Chapter 2 established the pressure point. Hard optimization can be encoded in
energy language, but encoding does not by itself produce an algorithm. To say
anything precise, one must fix a computational model and a resource measure.

This chapter fixes that baseline in the circuit and query setting. It develops
the finite-dimensional quantum formalism used throughout the thesis, defines the
computational model, and establishes the unstructured-search frontier that later
chapters must match or explain. The route is deliberate. First come states,
measurement, and Hamiltonian dynamics. Then come circuits, oracles, and
$\mathrm{BQP}$. Then comes Grover's algorithm as geometry in a two-dimensional
subspace, followed by the BBBV lower bound that proves the quadratic speedup is
optimal. Only after this baseline is closed does it make sense to turn to
adiabatic computation.

\section{Quantum Mechanics Essentials}
\label{sec:ch3-states-measurement}

The modern theory of quantum computation begins with a modeling tension. Feynman
argued that generic quantum dynamics appears classically expensive to simulate
\cite{Feynman1982}. Deutsch then gave a universal model of quantum computation
\cite{Deutsch1985}. Shor and Grover later turned that modeling claim into explicit
complexity separations \cite{Shor1997, Grover1996}.

For this thesis, the operational lesson is simple. Any complexity claim is
meaningful only after the computational model and its counted resources are fixed.
Changing the model changes which lower bounds are provable.

A pure state of a finite-dimensional closed quantum system is a unit vector
$\ket{\psi}$ in a complex Hilbert space $\mathcal{H}$
\cite{NielsenChuang2010, SakuraiNapolitano2017}. A single qubit uses
\begin{equation}
\label{eq:ch3-qubit-space}
\mathcal{H}_1 = \mathbb{C}^2 = \mathrm{span}\{\ket{0},\ket{1}\}.
\end{equation}
For $n$ qubits,
\begin{equation}
\label{eq:ch3-nqubit-space}
\mathcal{H}_n = (\mathbb{C}^2)^{\otimes n}, \qquad \dim \mathcal{H}_n = N=2^n.
\end{equation}
The computational basis is $\{\ket{x}:x\in\{0,1\}^n\}$, and every pure state has
an amplitude expansion
\begin{equation}
\label{eq:ch3-state-expansion}
\ket{\psi} = \sum_{x\in\{0,1\}^n} \alpha_x\ket{x},
\qquad
\sum_x |\alpha_x|^2 = 1.
\end{equation}
Dirac and von Neumann gave the canonical operator-state formalism used here
\cite{Dirac1930, vonNeumann1955}.

Composite systems introduce non-factorizable states. A bipartite state
$\ket{\psi_{AB}}$ is entangled if no pair $(\ket{\psi_A},\ket{\psi_B})$ exists
with
\begin{equation}
\label{eq:ch3-entanglement-definition}
\ket{\psi_{AB}} = \ket{\psi_A}\otimes\ket{\psi_B}.
\end{equation}
Entanglement by itself is not a speedup theorem, but it expands the reachable
space of correlations and therefore the algorithmic design space.

Global phase is physically irrelevant:
\begin{equation}
\label{eq:ch3-global-phase}
\ket{\psi} \equiv e^{i\phi}\ket{\psi}.
\end{equation}
Relative phase is not. Interference depends on relative phase, and interference is
where quantum algorithms get non-classical behavior.

Measurement is represented by orthogonal projectors $\{P_m\}$ with
$P_mP_{m'}=\delta_{m,m'}P_m$ and $\sum_m P_m = I$. The Born rule gives
\begin{equation}
\label{eq:ch3-born-rule}
\Pr[m] = \bra{\psi}P_m\ket{\psi}.
\end{equation}
For rank-one measurement onto $\ket{\phi}$,
\begin{equation}
\label{eq:ch3-born-rank-one}
\Pr[\phi] = |\langle\phi|\psi\rangle|^2.
\end{equation}
The post-measurement state for outcome $m$ is
$P_m\ket{\psi}/\sqrt{\Pr[m]}$, when $\Pr[m]>0$.

Mixed states are represented by density operators
\begin{equation}
\label{eq:ch3-density-operator}
\rho = \sum_j p_j \ket{\psi_j}\bra{\psi_j},
\qquad p_j\ge 0,\quad \sum_j p_j=1,
\end{equation}
with measurement rule
\begin{equation}
\label{eq:ch3-born-density}
\Pr[m]=\Tr(P_m\rho).
\end{equation}
The thesis mostly analyzes pure-state dynamics, but later adiabatic error bounds
are naturally expressed on $\rho(s)$.

Eqs.~\eqref{eq:ch3-state-expansion}--\eqref{eq:ch3-born-rank-one} already expose a
critical computational constraint. Amplitudes are not directly readable classical
data. A measurement returns one sampled outcome, not the full amplitude table.
Algorithmic advantage therefore requires controlled interference before
measurement.

\section{Hamiltonians and Time Evolution}
\label{sec:ch3-hamiltonians-dynamics}

A Hamiltonian $H$ is Hermitian. Hermiticity guarantees real eigenvalues and an
orthogonal eigenspace decomposition. In non-degenerate form,
\begin{equation}
\label{eq:ch3-spectral-nondegenerate}
H = \sum_j \lambda_j \ket{\phi_j}\bra{\phi_j},
\end{equation}
where each eigenpair satisfies
\begin{equation}
\label{eq:ch3-eigenpair-definition}
H\ket{\phi_j}=\lambda_j\ket{\phi_j},
\end{equation}
and in projector form,
\begin{equation}
\label{eq:ch3-spectral-projectors}
H = \sum_k E_k P_k,
\end{equation}
where $P_k$ projects onto the eigenspace at energy $E_k$.
The expected energy in state $\ket{\psi}$ is
\begin{equation}
\label{eq:ch3-energy-expectation}
\langle H \rangle_\psi = \bra{\psi}H\ket{\psi}.
\end{equation}
For any sufficiently regular scalar function $f$, spectral calculus gives
\begin{equation}
\label{eq:ch3-spectral-calculus}
f(H)=\sum_k f(E_k)P_k,
\end{equation}
which yields $e^{-itH}=\sum_k e^{-itE_k}P_k$ as the time-evolution operator in
the eigenbasis.

For optimization language, order distinct energies as
$E_0<E_1<\cdots$. The ground energy is $E_0$, and the first spectral gap is
\begin{equation}
\label{eq:ch3-static-gap}
\Delta = E_1 - E_0.
\end{equation}
If ground space is degenerate with dimension $d_0$, write its projector as
\begin{equation}
\label{eq:ch3-ground-projector}
P_0 = \sum_{j=1}^{d_0} \ket{\phi_{0,j}}\bra{\phi_{0,j}}.
\end{equation}
Later adiabatic success statements are naturally written as overlap with $P_0$,
not with a single selected vector.

Closed-system dynamics follows the Schrodinger equation (units $\hbar=1$). For
constant $H$,
\begin{equation}
\label{eq:ch3-schrodinger-static}
i\frac{d}{dt}\ket{\psi(t)} = H\ket{\psi(t)},
\end{equation}
with solution
\begin{equation}
\label{eq:ch3-unitary-solution}
\ket{\psi(t)} = U(t)\ket{\psi(0)}, \qquad U(t)=e^{-itH}.
\end{equation}
Unitarity,
\begin{equation}
\label{eq:ch3-unitary-condition}
U^\dagger U = I,
\end{equation}
is exactly the algebraic statement that total probability is preserved during
closed-system evolution.
For time-dependent $H(t)$,
\begin{equation}
\label{eq:ch3-schrodinger-time-dependent}
i\frac{d}{dt}\ket{\psi(t)} = H(t)\ket{\psi(t)},
\end{equation}
with formal propagator
\begin{equation}
\label{eq:ch3-time-ordered-propagator}
\ket{\psi(t)} = \mathcal{T}\exp\!\left(-i\int_0^t H(\tau)\,d\tau\right)\ket{\psi(0)}.
\end{equation}

Two remarks now prevent confusion later. First, adding $cI$ to $H$ shifts all
energies but does not change eigenvectors. Second, this shift contributes only a
global phase under unitary evolution. Measured probabilities are unchanged.
Normalization choices such as setting a ground energy to zero are therefore
computationally harmless.

Finally, distinguish static and path-dependent gaps carefully. Eq.~\eqref{eq:ch3-static-gap}
defines a static spectral gap of one Hamiltonian. In adiabatic evolution,
$H=H(s)$ varies with schedule parameter $s$, and the relevant quantity becomes
$g(s)=\lambda_1(s)-\lambda_0(s)$ along the full path.

\section{Why Quantum Helps}
\label{sec:ch3-speedup-mechanisms}

At this point, the right question is not whether quantum mechanics is
``powerful'' in general. The right question is which mechanism a specific
algorithm exploits and how that mechanism is certified by a complexity bound.
For unstructured search, the mechanism is constructive interference in a
two-dimensional invariant subspace. For Shor's algorithm, it is Fourier
structure over periodicity. Different speedups come from different structures.

Superposition alone is not a speedup theorem. Entanglement alone is not a
speedup theorem. Both enlarge state geometry, but complexity gains still require
a unitary process that maps problem structure to measurable amplitude
separation. This is why query lower bounds are indispensable. They separate
what is physically expressible from what is information-theoretically learnable
with bounded resources.

Tunneling is often discussed in optimization contexts, especially for adiabatic
and annealing dynamics. It can help in specific barrier profiles, but broad
claims of generic optimization advantage from tunneling alone are not proved.
This thesis therefore tracks only quantities that support explicit theorems:
oracle/query complexity in this chapter, then spectral gaps, schedules, and
adiabatic error bounds in Chapters 4--9.

\section{Circuits and Query Complexity}
\label{sec:ch3-circuit-model}

In the gate model, an algorithm is a uniform family of circuits over a universal
gate set. Barenco et al.~established standard finite universal gate constructions
\cite{Barenco1995}, and Bernstein-Vazirani formalized complexity in terms of
uniform circuit families \cite{BernsteinVazirani1997}. Uniformity means there is
a classical polynomial-time procedure that outputs the circuit for input length
$n$. Complexity is then measured by resources such as size, depth, and ancilla
count as functions of $n$.

The choice of finite universal gate set does not change the complexity class.
Solovay-Kitaev type compilation guarantees polylogarithmic overhead in target
precision for approximating unitaries, so $\mathrm{BQP}$ is robust under standard
gate-set choices \cite{KitaevShenVyalyi2002, DawsonNielsen2006}.

A schematic $T$-query quantum algorithm for oracle $f$ has the interleaved form
\begin{equation}
\label{eq:ch3-query-interleaving}
\ket{\psi^{(T)}} = U_T O_f U_{T-1} O_f \cdots O_f U_1 O_f U_0\ket{0^m},
\end{equation}
where $U_j$ are input-independent unitaries and $O_f$ carries the instance
information. This is the fixed-oracle primitive. Optimization algorithms that use
threshold updates run this primitive repeatedly with different oracles
$O_{f_\tau}$ across outer-loop iterations.

For a function family $f_n$, bounded-error quantum query complexity $Q_2(f_n)$ is
the minimum number of oracle calls in any algorithm of the form
Eq.~\eqref{eq:ch3-query-interleaving} that succeeds with probability at least
$2/3$ on every valid input of size $n$. Classical randomized query complexity
$R_2(f_n)$ is defined analogously with randomized decision trees
\cite{arora2009computational, NielsenChuang2010}.

This separation is deliberate. Query complexity isolates information acquisition.
Gate complexity measures full implementation cost. The thesis uses this query
frontier as a baseline because Chapter 5 opens with unstructured optimization in
that exact black-box sense.

The complexity class $\mathrm{BQP}$ contains promise problems solvable by uniform
polynomial-size quantum circuits with bounded two-sided error. This definition is
due to Bernstein-Vazirani \cite{BernsteinVazirani1997}; Watrous and Arora-Barak
provide standard modern treatments \cite{Watrous2009, arora2009computational}. A
promise problem is a pair $(L_{\mathrm{yes}},L_{\mathrm{no}})$ with
$L_{\mathrm{yes}}\cap L_{\mathrm{no}}=\varnothing$. A
$\mathrm{BQP}$ algorithm must accept $x\in L_{\mathrm{yes}}$ with probability at
least $2/3$ and reject $x\in L_{\mathrm{no}}$ with probability at least $2/3$.
Standard repetition and majority vote reduce error exponentially, so the constant
$2/3$ is conventional.
Promise formulations are not a technicality. They let us state complexity on the
physically relevant input domain while leaving behavior unconstrained off-promise.

For orientation in the complexity landscape,
$\mathrm{P}\subseteq\mathrm{BQP}\subseteq\mathrm{PSPACE}$, with a standard route
through $\mathrm{BQP}\subseteq\mathrm{PP}\subseteq\mathrm{PSPACE}$
\cite{BernsteinVazirani1997, Watrous2009}. These inclusions are known; strictness
relations remain open.

One distinction is worth fixing now. $\mathrm{BQP}$ is a decision-class notion
for promise problems. Minimum finding is an optimization task. The bridge is
threshold reduction: optimization is solved by repeated decision queries of the
form ``is there any item with cost at most $\tau$?'' This reduction is the
reason query-optimal search bounds transfer directly to unstructured minimum
finding.
Operationally, the decision primitive is
\[
D_\tau(C) = \mathbf{1}\!\left[\exists x,\ C(x)\le \tau\right].
\]
Optimization then becomes threshold management plus calls to $D_\tau$.

\section{Unstructured Optimization in the Oracle Model}
\label{sec:ch3-oracle-unstructured}

For a Boolean oracle $f:\{0,1\}^n\to\{0,1\}$, the standard query action is
\begin{equation}
\label{eq:ch3-standard-oracle}
O_f\ket{x,y} = \ket{x, y\oplus f(x)}.
\end{equation}
An equivalent phase-oracle form is
\begin{equation}
\label{eq:ch3-phase-oracle}
O_f\ket{x} = (-1)^{f(x)}\ket{x}.
\end{equation}
These forms are interconvertible with one ancilla, so lower and upper bounds
transfer between them.

Unstructured search asks for a marked input in a set
$W\subseteq\{0,1\}^n$ of size $d_0\ge 1$, given only oracle access.
Unstructured minimum finding asks for
$x^\star\in\arg\min_x C(x)$ for a black-box cost function
$C:\{0,1\}^n\to\mathbb{R}$ with no promised exploitable structure.
By contrast, structured optimization provides additional representation-level
regularity, such as locality, sparsity, algebraic constraints, or graph
symmetry, that can be exploited algorithmically. The unstructured model used for
Grover and BBBV deliberately removes that advantage.

The word ``unstructured'' names the information model, not the computational
paradigm. In the circuit model, one accesses the instance through oracle queries.
In adiabatic optimization, one accesses it through the diagonal energy map
$z\mapsto E_z$ of $H_z$, with no additional promised regularity such as locality,
sparsity, or exploitable algebraic pattern. The two models
therefore enforce the same absence of exploitable structure through different
physical primitives.
In worst-case formulations, no generic advantage can rely on basis labels
themselves. Only permutation-invariant summaries, such as the number of marked
states, are available to all algorithms \cite{bennett1997strengths}.

The reductions between these tasks are explicit. Search reduces to optimization by
\begin{equation}
\label{eq:ch3-search-to-min}
C_f(x) = 1-f(x),
\end{equation}
so minimizers are exactly marked items. Optimization reduces to threshold search
with
\begin{equation}
\label{eq:ch3-min-to-search}
f_\tau(x) = \mathbf{1}[C(x)\le \tau].
\end{equation}
Durr and Hoyer use this threshold view with Grover subroutines to obtain
minimum finding in
\begin{equation}
\label{eq:ch3-durr-hoyer-runtime}
\Theta\!\left(\sqrt{\frac{N}{d_0}}\right)
\end{equation}
queries up to failure-amplification factors \cite{durr1996quantum, BBHT1998,
BrassardHoyerMoscaEtAl2002}.

This is the baseline Chapter 5 invokes. In the circuit-query model, unstructured
optimization already has a sharp exponent.

\section{Grover as Geometry}
\label{sec:ch3-grover-geometry}

Let $W$ be the marked set with size $d_0$. Define normalized marked and unmarked
superpositions
\begin{equation}
\label{eq:ch3-marked-unmarked-states}
\ket{w} = \frac{1}{\sqrt{d_0}}\sum_{x\in W}\ket{x},
\qquad
\ket{r} = \frac{1}{\sqrt{N-d_0}}\sum_{x\notin W}\ket{x}.
\end{equation}
The uniform superposition is
\begin{equation}
\label{eq:ch3-uniform-state-decomposition}
\ket{s}=\frac{1}{\sqrt{N}}\sum_x\ket{x}
=\sin\theta\,\ket{w}+\cos\theta\,\ket{r},
\qquad
\sin\theta=\sqrt{\frac{d_0}{N}}.
\end{equation}
The dynamics closes on
$\mathrm{span}\{\ket{w},\ket{r}\}$.

Grover's iterate is a product of two reflections,
\begin{equation}
\label{eq:ch3-grover-iterate}
G=(2\ket{s}\bra{s}-I)(I-2\ket{w}\bra{w}),
\end{equation}
which is a rotation by angle $2\theta$ in that plane. In basis
$(\ket{w},\ket{r})$,
\begin{equation}
\label{eq:ch3-grover-matrix}
G=
\begin{pmatrix}
\cos 2\theta & \sin 2\theta\\
-\sin 2\theta & \cos 2\theta
\end{pmatrix}.
\end{equation}
Hence
\begin{equation}
\label{eq:ch3-grover-k-step}
G^k\ket{s} = \sin((2k+1)\theta)\ket{w}+\cos((2k+1)\theta)\ket{r},
\end{equation}
so success probability is
$\sin^2((2k+1)\theta)$.

The same operation has a circuit decomposition: one oracle reflection and one
diffusion reflection per iteration. Writing
\begin{equation}
\label{eq:ch3-grover-oracle-diffusion}
O_w = I-2\ket{w}\bra{w},
\qquad
D = 2\ket{s}\bra{s}-I,
\end{equation}
the iterate is $G=DO_w$. The algorithm is therefore a repeated application of a
fixed two-block circuit.
The diffusion operator itself can be written as
\begin{equation}
\label{eq:ch3-diffusion-hadamard-form}
D = H^{\otimes n}\left(2\ket{0^n}\bra{0^n}-I\right)H^{\otimes n},
\end{equation}
which is the form implemented in gate-level circuits.

Choosing
\begin{equation}
\label{eq:ch3-grover-optimal-k}
k^*\approx \left\lfloor\frac{\pi}{4\theta}-\frac{1}{2}\right\rfloor
\end{equation}
maximizes success. For $d_0\ll N$, $\theta\sim\sqrt{d_0/N}$, giving
\begin{equation}
\label{eq:ch3-grover-runtime}
Q_{\mathrm{Grover}}=\Theta\!\left(\sqrt{\frac{N}{d_0}}\right)
\end{equation}
queries \cite{Grover1996, BBHT1998, BrassardHoyerMoscaEtAl2002}.
When $d_0$ is unknown, BBHT-style randomized iteration schedules retain the same
asymptotic scaling in expectation while preserving bounded-error success
\cite{BBHT1998}.
For the one-solution case $d_0=1$, this reduces to
$k^*=\lfloor(\pi/4)\sqrt{N}\rfloor+O(1)$ queries.

The geometric point matters for this thesis beyond Chapter 3. A large Hilbert
space can still reduce to an effective two-level dynamics controlling runtime.
Chapter 5 uses the same structural motif at the avoided crossing of adiabatic
interpolation.

\section{Why Grover Is Optimal}
\label{sec:ch3-bbbv-optimality}

The query lower bound of Bennett, Bernstein, Brassard, and Vazirani (BBBV) states
that unstructured search requires
\begin{equation}
\label{eq:ch3-bbbv-lower-bound}
\Omega\!\left(\sqrt{\frac{N}{d_0}}\right)
\end{equation}
queries for bounded-error success \cite{bennett1997strengths}. For $d_0=1$, this
is $\Omega(\sqrt{N})$.

A useful proof intuition is hybrid distinguishability growth. Compare final states
under nearby oracles. Let $\ket{\psi_T^{(w)}}$ be the final state when the marked
set is $\{w\}$, and let $\ket{\psi_T^{(0)}}$ be the final state for the null
oracle (no marked item). One query changes average distinguishability by at most a
controlled amount, so after $T$ queries,
\begin{equation}
\label{eq:ch3-hybrid-bound}
\frac{1}{N}\sum_{w\in[N]}
\left\|\ket{\psi_T^{(w)}}-\ket{\psi_T^{(0)}}\right\|_2
\leq \frac{2T}{\sqrt{N}}.
\end{equation}
When $T\ll \sqrt{N}$, the average distance remains $o(1)$, so the final states do
not separate enough for any measurement to identify the marked oracle with
constant bias. Constant success therefore needs $T=\Omega(\sqrt{N})$.
The same argument with $d_0$ marked items yields
$\Omega(\sqrt{N/d_0})$ up to constants. Nayak and Wu provide a complementary
polynomial-method perspective on related bounds \cite{NayakWu1999}.

Combining Eq.~\eqref{eq:ch3-grover-runtime} with
Eq.~\eqref{eq:ch3-bbbv-lower-bound} gives the exact frontier in the unstructured
query model:
\begin{equation}
\label{eq:ch3-optimal-frontier}
Q_{\mathrm{search}} = \Theta\!\left(\sqrt{\frac{N}{d_0}}\right).
\end{equation}
This is not merely a known algorithm. It is a closed asymptotic frontier.

The same exponent appears in continuous-time query models. Farhi, Goldstone, and
Gutmann proved an $\Omega(\sqrt{N/d_0})$ lower bound for unstructured adiabatic
search with a rank-one projector driver \cite{farhi2008fail}. This is the formal
bridge to Chapter 4. Changing the implementation from discrete queries to
Hamiltonian evolution does not remove the unstructured lower-bound barrier.

\section{Scope and Transition}
\label{sec:ch3-scope-transition}

This chapter worked in an ideal closed-system complexity model. Real devices are
open systems and suffer decoherence and control noise \cite{Zurek2003}. Fault-tolerant
frameworks address this in principle through encoding and active correction
\cite{Kitaev1997, NielsenChuang2010}. Those engineering layers matter in practice,
but they are not the analytical target of Chapters 4--9.

The next step keeps the optimization objective fixed and changes only the model.
In the circuit-query setting, unstructured optimization is already pinned down by
Eq.~\eqref{eq:ch3-optimal-frontier}. In adiabatic computation, the same objective
is pursued by continuous Hamiltonian evolution, and runtime is governed by
pathwise spectral geometry rather than query count alone. Chapter 4 asks what
this model shift buys and what it demands, once schedule design and gap control
become algorithmic resources in their own right. That is the precise sense in
which Chapter 5 can open by saying that, in the circuit model, unstructured
optimization is already understood.
